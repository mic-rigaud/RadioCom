\chapter{Estimation de la réponse fréquentielle du canal de propagation dans les structures d'égalisation DFVE}

\paragraph{}
Dans ce chapitre, nous supposons ne plus connaître la réponse fréquentielle du canal, et donc nous ne connaissons plus les matrices qui interviennent dans les structures d'égalisation DFVE. Le but est de les approcher grâce au principe de l'algorithme LMS (Algorithme du gradient stochatisque).

\section{Création du signal et modélisation du canal de propagation}
\paragraph{}
Dans notre signal, nous avons ajouté des pilotes. Les pilotes sont des états connus qui servent à estimer le canal à la réception. Dans notre cas, l'état des pilotes a été pris à $1+j$, et ces pilotes ont été insérer tout les 3 symboles OFDM, sur nos 4 porteuses. Figure ~\ref{etatavecpilotes}, nous pouvons voir le signal que nous voulons envoyer après modulation $\pi/4$-QPSK avec les pilotes. 
\paragraph{}
\vspace{1\baselineskip}
\begin{figure}[!h]
  \centering
  \includegraphics[scale=0.6]{emissionavecpilote.png}
  \caption{Signal après modulation QPSK et avec ajout des pilotes }
	\label{etatavecpilotes} 
\end{figure}
\vspace{10\baselineskip}

\paragraph{}
Le canal de propagation fût modélisé comme dans le chapitre précédent, et Figure ~\ref{sanstraitement}, on peut voir le signal reçu à la réception avant les traîtements.
\paragraph{}
\vspace{1\baselineskip}
\begin{figure}[!h]
  \centering
  \includegraphics[scale=0.7]{sanstraitement.png}
  \caption{Signal reçu par le récepteur}
	\label{sanstraitement} 
\end{figure}
\vspace{2\baselineskip}


\section{Egalisation avec estimation de la réponse du canal}

\paragraph{}
Dans les deux structures, il falait un point de départ pour les matrices. Pour simuler la convergence vers nos bonnes matrices, nous avons fausser celle de départ en ajoutant des valeurs aux coefficients de $H0$ et de $H1$. De plus, nous avons un pas d'adaptation $\mu$ de 0.01. De notre interpretation, le pas d'adaptation petit ralentit l'estimation de la réponse du canal, mais on sera plus précis. Si notre pas est grand, on sera rapide, mais moins précis.

\subsection{Structure fréquentielle d'égalisation DFVE}
\paragraph{}
Dans notre chaîne de réception, lors de la réception de nos pilotes, nous réestimons nos matrices $P0$ et $P1$ avec les formules 12a et 12b de \cite{sujet}. Une fois les états du signal estimés, nous enlevons les pilotes, et nous obtenons le diagramme de constellation visible sur la Figure ~\ref{etatsanspilote}.

\paragraph{}
\vspace{1\baselineskip}
\begin{figure}[!h]
  \centering
  \includegraphics[scale=0.6]{etatsanspilotes.png}
  \caption{Etats du signal estimés après extraction des pilotes dans la structure fréquentielle }
	\label{etatsanspilote} 
\end{figure}

\paragraph{}
Après comparaison avec le signal emis, nous obtenons 0\% d'erreurs sur l'ensemble du signal.


\subsection{Structure temporelle d'égalisation DFVE}

\paragraph{}
Pour la structure temporelle, on estime Q0 et Q1 dans la boucle, mais on fait une estimation sous optimale sur Q1 afin de réduire la complexité. A cause de cette estimation, nous avons toujours plus d'erreurs que dans la structure fréquentielle, comme 1.12\% où l'on trouvait 0\% pour la structure fréquentielle.
\paragraph{}
Afin de mieux illustrer le signal après estimation, nous pouvons voir sur la Figure ~\ref{avecPilote} l'estimation des états par la structure temporelle, avant extraction des pilotes.
\paragraph{}
\paragraph{}
\vspace{1\baselineskip}
\begin{figure}[!h]
  \centering
  \includegraphics[scale=0.6]{avecPilote.png}
  \caption{Etats du signal estimés dans la structure temporelle avant extraction des pilotes }
	\label{avecPilote} 
\end{figure}
\vspace{1\baselineskip}
\paragraph{}
Afin de tester la convergence de notre méthode, nous avons tester un code où nous n'ajoutons pas de bruit complexe dans le canal. Nous mettons beaucoup d'erreurs dans les matrices H0 et H1 initiales, donc dans Q0, Q1, P0 et P1. Dans un premier test, nous gardons cette matrice, et ne la régulons pas. Nous obtenons 26\% d'erreurs dans les deux structures. Ensuite nous relançons l'algorithme avec l'estimation continue des matrices. Nous avons 19\% d'erreurs sur la structure fréquentielle, et 20\% sur la structure temporelle. C'est normal d'avoir un fort taux d'erreurs ici, car il faut le temps que le récepteur converge vers la bonne estimation. Ce qui est important est que nous voyons que nous sommes meilleurs avec la convergence vers la bonne solution grâce à l'algorithme LMS, car on fini par bien estimer notre canal. C'est ce que l'on peut voir sur la Figure ~\ref{convergence}. On commence par des estimations erronnées, et à la fin, on peut voir la convergence vers les bons états.
\paragraph{}
\vspace{1\baselineskip}
\begin{figure}[!h]
  \centering
  \includegraphics[scale=0.5]{convergence.png}
  \caption{Convergence vers les bons états grâce à l'algorithme LMS }
	\label{convergence} 
\end{figure}
\paragraph{}
Nous venons de tester les estimations de la réponse du canal de propagation à travers les estimations de matrices. Nous avons constater un compromis entre complexité de calcul et efficacité de l'éstimation. En effet, la structure temporelle diminue fortement la complexité calculatoire, mais il y a plus d'approximation dans l'éstimation du canal, ce qui nous fait perdre de l'éfficacité dans l'éstimation final du signal émis. Il faut donc trouver un juste milieu dans ce que l'on peut corriger grâce à un codage canal, et la puissance et le temps de calcul minimum.



