
\chapter{Justification des relations d'égalisation DFVE}

Dans le document \cite{sujet}, plusieurs équations méritent d'être discutées.

\section{Équation (2) et (4)}
\label{sec:24}

\paragraph{Énonce :}

Soit $x_i$ et $y_i$ les vecteurs formés des N échantillons des $j^{ème}$
symbole émis et reçus, $b_j$ le vecteur de bruit et $H_0$ et $H_1$ les matrices
triangulaires définies par :

$H_0=
\begin{bmatrix}
  h_0 & 0 & \cdots & \cdots & \cdots & 0 \\
  h_1 & h_0 & 0 & \cdots  & \cdots & \vdots \\
  \vdots & & \ddots & \ddots &  & \vdots \\
  h_{N-1} & h_{N-2} & \cdots & \cdots & h_1 & h_0

\end{bmatrix}
$  $H_1=
\begin{bmatrix}
  0 & h_{N-1} & h_{N-2} & \cdots & \cdots & h_1 \\
  0 & 0 & h_{N-1} & h_{N-2} &  & h_2 \\
  \vdots & 0 & \ddots & \ddots &   & \vdots \\
  \vdots &  &  & \ddots  & \ddots & h_{N-1} \\
  0 & \cdots & \cdots & \cdots & 0 & 0 \\


\end{bmatrix}
$

Ou les coefficients $h_0,h_1,\cdots,h_{N-1}$ représentent les échantillons de la
réponse du canal.

Alors,

$ y_j = H_0*x_j+H_1*x_{j-1}+b_j$
~\\

Et en FFT, on peut écrire:

$ Y_j = C_{dft}(H_0*C_{DFT}^{-1}*X_j + H_1*C_{DFT}^{-1}*X_{j-1}) + B_j$

\paragraph{Justification :}

Considérons une séquence de $N$ données $c_{j,0},c_{j,1},\cdots,c_{j,N-1}$ pour le symbole
OFDM $j$. Chaque donnée $c_{j,k}$ module un signal à la fréquence $f_k$. Donc le
signal sur cette fréquence s'écrit sous forme complexe: $c_{j,k}*e^{2j\pi f_kt}$

Le signal $x_j(t)$ total transmis sur le symbole OFDM $j$ est donc: \\
$x_j(t)=\sum_{k=0}^{N-1}c_{j,k}e^{2j\pi f_kt}$

Or le multiplexage est orthogonal donc $f_k=f_0+\frac{1}{T}$

Ainsi, en posant $\Delta f = \frac{1}{T}$, on obtient: \\
$x_j(t)=e^{2j\pi f_0t}\sum_{k=0}^{N-1}c_{j,k}e^{2j\pi k\Delta ft}$

En ramenant le signal en bande de base on a:\\
$x_j(t)=\sum_{k=0}^{N-1}c_{j,k}e^{2j\pi k\Delta ft}$
~\\

De même, le signal parvenant au récepteur s'écrit:\\
$y_j(t)=\sum_{k=0}^{N-1}c_{j,k}H_k(t)e^{2j\pi (f_0+\frac{k}{T})t}$, avec $H_k(t)$ la
fonction de transfert du canal autour de la fréquence $f_k$ et au temps $t$.

Si on discrétise le symbole on pourra écrire que: \\
$\forall n \in [0,N-1], y_j(n) = h_0*x_j(n)$, avec $h_0$ la fonction de
transfert du canal direct.

Seulement, le canal subit des multi-trajets et du bruit. Donc il va falloir
ajouter les états précédents avec une fonction de transfert différente qui
correspondra à la fonction de transfert du canal pour un certain echo.

Ainsi,

$\forall n \in [0,N-1], y_j(n) = h_0*x_j(n) + h_1*x_j(n-1) + \cdots
+h_{n}*x_j(0) + h_{n+1}*x_{j-1}(N-1) + \cdots + h_{N-1}*x_{j-1}(n+1) + b_j$

En écriture matricielle, on obtient donc:

$ y_j = H_0*x_j+H_1*x_{j-1}+b_j$, avec les matrices triangulaires définies précédemment.

\section{Équation (6) et (8)}

\paragraph{Énonce :}


L'égalisation doit fournir une estimation de $\widehat{X_j}$. Dans le domaine
fréquentiel l'équation est:
~\\

$\widehat{X_j}= P_0*Y_j+P_1*\tilde{X_j}$

Les matrices $P_0$ et $P_1$ suivant le critère de ZF (Zero forcing) sont:

$P_0^{ZF} = C_{DFT}*H_0^{-1}*C_{DFT}^{-1}$

$P_1^{ZF} = - C_{DFT}*H_0^{-1}*H_1*C_{DFT}^{-1}$




%%% Local Variables:
%%% mode: latex
%%% TeX-master: "../rapport_de_base"
%%% End:
