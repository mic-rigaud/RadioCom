
\chapter{Justification des relations d'égalisation DFVE}

Dans le document \cite{sujet}, plusieurs équations méritent d'être discutées.

\section{Équation (2) et (4)}
\label{sec:ISI}

\textbf{Énonce} :

Soit $x_i$ et $y_i$ les vecteurs formés des N échantillons des $j^{ème}$
symbole émis et reçus, $b_j$ le vecteur de bruit et $H_0$ et $H_1$ les matrices
triangulaires définies par :

$H_0=
\begin{bmatrix}
  h_0 & 0 & \cdots & \cdots & \cdots & 0 \\
  h_1 & h_0 & 0 & \cdots  & \cdots & \vdots \\
  \vdots & & \ddots & \ddots &  & \vdots \\
  h_{N-1} & h_{N-2} & \cdots & \cdots & h_1 & h_0

\end{bmatrix}
$  $H_1=
\begin{bmatrix}
  0 & h_{N-1} & h_{N-2} & \cdots & \cdots & h_1 \\
  0 & 0 & h_{N-1} & h_{N-2} &  & h_2 \\
  \vdots & 0 & \ddots & \ddots &   & \vdots \\
  \vdots &  &  & \ddots  & \ddots & h_{N-1} \\
  0 & \cdots & \cdots & \cdots & 0 & 0 \\


\end{bmatrix}
$

Alors,

$ y_j = H_0*x_j+H_1*x_j-1+b_j$
~\\

Et en FFT, on peut écrire:

$ Y_j = C_{dft}(H_0*C_{DFT}^{-1}*X_j + H_1*C_{DFT}^{-1}*X_{j-1}) + B_j$


\section{Équation (6) et (8)}

\textbf{Énonce : }

L'égalisation doit fournir une estimation de $\widehat{X_j}$. Dans le domaine
fréquentiel l'équation est:
~\\

$\widehat{X_j}= P_0*Y_j+P_1*\tilde{X_j}$

Les matrices $P_0$ et $P_1$ suivant le critère de ZF (Zero forcing) sont:

$P_0^{ZF} = C_{DFT}*H_0^{-1}*C_{DFT}^{-1}$

$P_1^{ZF} = - C_{DFT}*H_0^{-1}*H_1*C_{DFT}^{-1}$




%%% Local Variables:
%%% mode: latex
%%% TeX-master: "../rapport_de_base"
%%% End:
