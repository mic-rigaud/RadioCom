\documentclass[a4paper, 11pt, oneside, oldfontcommands]{memoir}

%%%%% Packages %%%%%
\usepackage{lmodern}
\usepackage{palatino}
\usepackage[T1]{fontenc}
\usepackage[utf8]{inputenc}
\usepackage[french]{babel}


%%%%%%%%%%%%%%%%%%%%  PACKAGE SECONDAIRE

%\usepackage{amstext,amsmath,amssymb,amsfonts} % package math
%\usepackage{multirow,colortbl}	% to use multirow and ?
%\usepackage{xspace,varioref}
\usepackage[linktoc=all, hidelinks]{hyperref}			% permet d'utiliser les liens hyper textes
\usepackage{float}				% permet d ajouter d autre fonction au floatant
%\usepackage{wrapfig}			% permet d avoir des image avec texte coulant a cote
%\usepackage{fancyhdr}			% permet d inserer des choses en haut et en bas de chaque page
\usepackage{microtype}			% permet d ameliorer l apparence du texte
\usepackage[explicit]{titlesec}	% permet de modifier les titres
\usepackage{graphicx}			% permet d utiliser les graphiques
\graphicspath{{./images/}}		% to say where are image
%\usepackage{eso-pic} 			% to put figure in the background
\usepackage[svgnames]{xcolor}	% permet d avoir plus de 300 couleur predefini
%\usepackage{array}				% permet d ajouter des option dans les tableaux
%\usepackage{listings}			% permet d ajouter des ligne de code
%\usepackage{tikz}				% to draw figure
%\usepackage{appendix}			% permet de faire les index
%\usepackage{makeidx}			% permet de creer les index
%\usepackage{fancyvrb}			% to use Verbatim
%\usepackage{framed}				% permet de faire des environnement cadre
%\usepackage{fancybox}			% permet de realiser les cadres
\usepackage{titletoc}			% permet de modifier les titres
%\usepackage{caption}
%\usepackage[a4paper, top=2cm, bottom=2cm]{geometry}
\usepackage{frbib}                      %permet d avoir une biblio francaise
\usepackage[babel=true]{csquotes}


\usepackage{graphicx}
\RequirePackage{pageGardeEnsta}	% permet d avoir la page de garde ensta

%\setcounter{secnumdepth}{2}		% permet d'augmenter la numerotation
%\setcounter{tocdepth}{2}		% permet d'augmenter la numerotation

%%%%%%%%%%%%%%%%%%  DEFINITION DES BOITES
\newcounter{rem}[chapter]

\newcommand{\remarque}[1]{\stepcounter{rem}\noindent\fcolorbox{OliveDrab}{white}{\parbox{\textwidth}{\textcolor{OliveDrab}{
\textbf{Remarque~\thechapter.\therem~:}}\\#1}}}

\newcounter{th}[chapter]

\newcommand{\theoreme}[2]{\noindent\fcolorbox{FireBrick}{white}{\stepcounter{th}
\parbox{\textwidth}{\textbf{\textcolor{FireBrick}{Théorème~\thechapter.\theth~:}}{\hfill \textit{#1}}\\#2}}}

\newcommand{\attention}[1]{\noindent\fcolorbox{white}{white}{\parbox{\textwidth}{\textcolor{FireBrick}{
\textbf{Attention !}}\\\textit{#1}\\}}}
%%%%%%%%%%%%%%%%%%%%%%%%%%%%%%%%%%%%%%%%%%%%%%%%%%%%%%%%%%%%%%%%%%%%%%%%%


%% INDEX %%%%%%%%%%%%%%%%%%%%%%%%%%%%%%%%%%%%%%%%%%%%%%%%%%%%
\makeindex

%%%%% Useful macros %%%%%
\newcommand{\latinloc}[1]{\ifx\undefined\lncs\relax\emph{#1}\else\textrm{#1}\fi\xspace}
\newcommand{\etc}{\latinloc{etc}}
\newcommand{\eg}{\latinloc{e.g.}}
\newcommand{\ie}{\latinloc{i.e.}}
\newcommand{\cad}{c'est-à-dire }
\newcommand{\st}{\ensuremath{\text{\xspace s.t.\xspace}}}

%%%% Definition des couleur %%%%

\newcommand\couleurb[1]{\textcolor{SteelBlue}{#1}}
\newcommand\couleurr[1]{\textcolor{DarkRed}{#1}}


%% number page style style %%%%%%%%%%%%%%%%%%%%%%%%%%%%%%%%%%%%%%%%%%%%%%%%%%%%%%

\pagestyle{plain}
%\pagestyle{empty}
%\pagestyle{headings}
%\pagestyle{myheadings}



%% chapters style %%%%%%%%%%%%%%%%%%%%%%%%%%%%%%%%%%%%%%%%%%%%%%%%%%%%%%
%% You may try several styles (see more in the memoir manual).

%\chapterstyle{veelo}
%\chapterstyle{chappell}
%\chapterstyle{ell}
%\chapterstyle{ger}
%\chapterstyle{pedersen}
%\chapterstyle{verville}
\chapterstyle{madsen}
%\chapterstyle{thatcher}


%%%%% Report Title %%%%%
\title{Egalisation vectorielle pour signaux OFDM}
\author{\textsc{Rigaud Michaël} et \textsc{Coulmy Thomas}}
\date{\today}
\doctype{Projet Radiocommunication}
\promo{Promotion 2017}
\etablissement{\textsc{Ensta} Bretagne\\2, rue François Verny\\
  29806 \textsc{Brest} cedex\\\textsc{France}\\Tel +33 (0)2 98 34 88 00\\ \url{www.ensta-bretagne.fr}}
\logoEcole{\includegraphics[height=4.2cm]{logo_ENSTA_Bretagne_Vertical_CMJN}}



%%%%%%%%%%%%%%%%%% DEBUT DU DOCUMENT
\begin{document}

\maketitle
\thispagestyle{empty}
\newpage

\tableofcontents


%%%%%%%%%%%%%%%%% INTRODUCTION

\chapter*{Introduction}
\addcontentsline{toc}{chapter}{Introduction}
Avant de répondre précisément aux questions du projet donné par M ROSTAING, nous
souhaitons tout d’abord expliquer quelques problématiques des communications
sans-fils et ce qu’est le principe de l’OFDM () dans cette introduction.
Ensuite, dans les deux premières questions, nous nous intéresserons au protocole
OFDM avec intervalle de garde entre symboles. Puis, pour aller plus loin, nous
étudierons l’article\cite{sujet} avec des simulations MATLAB afin de mieux comprendre.
Ce dernier consiste à mettre en place un système permettant de se passer de
l’intervalle de garde, et ainsi, de ne pas perdre de débit à cause des temps
d’attentes entre symboles.
\section*{Problèmes généraux des transmissions de données sans-fils}
Nous allons présenter ici les deux principaux problèmes rencontrés lors du
passage du signal transmis dans le canal de propagation. Ceux-ci sont liés à la
réponse fréquentielle du canal de propagation, mais ont des phénomènes physiques
différents.

Le premier problème est l’interférence entre symbole. Cela est dû à la
dispersion des symboles dans le temps lorsque nous en envoyons plusieurs à la suite.

L’autre problème est l’affaiblissement par multi-trajets, aussi appelé Fading.
Cela arrive lorsque le même signal à l’émissions parcours des trajets différents
avec réflexions et diffractions, puis arrive sur le récepteur avec un décalage
dans le temps et des variations de phases par rapport au signal reçu en trajet direct.

\section*{Les signaux OFDM}

\begin{figure}[!h]
  \centering
  \includegraphics[width=\textwidth]{Frequence_time.png}
  \caption[Temps-Frequence]{Temps-Frequence: representation d un signal OFDM}
  \label{fig:tempsFreq}
\end{figure}

Comme nous pouvons le voir, les signaux OFDM résultent d’une modulation multi-porteuses. C’est-à-dire que nous répartissons l’information sur une bande de fréquence, autour de plusieurs porteuses de fréquences centrales également réparties. Puis, à chaque sous-fréquences porteuses, on envoie des symboles répartis dans le temps espacé par des intervalles de garde.



\paragraph{}


\newpage
%%%%%%%%%%%%%%%%%%%%%%%%

\chapter{Egalisation OFDM}

L’égalisation sert à réduire fortement, voir annuler, les interférences dues au
multi-trajets dans le canal de propagation. Dans le domaine temporel, elle se
fait en cherchant les coefficients d’atténuation modélisant l’effet du canal.
Mais, dans le cas de transmission à haut débit, nous avons trop de recouvrement
entre symbole à cause des retards lors de la réception des différents
multi-trajets, ainsi le système devient complexe et donc le coût des terminaux
devient élevé.

L’idée de l’égalisation OFDM est de transformer l’égalisation faite dans le
domaine temporel dans le domaine fréquentiel. En passant dans le domaine
fréquentiel, et en envoyant le signal sur plusieurs porteuses, on est capable
d’évaluer la réponse fréquentielle du canal de propagation sur la bande de
fréquence du signal total. Ce qui est important, c’est que la bande associée à
chaque sous-porteuses doit être dans une zone de cohérence du canal,
c’est-à-dire que la fonction de transfert du canal soit à peu près plat dans la
bande.  Sur l’image ci-dessous, on peut comprendre comment on estime la réponse
fréquentielle du canal par morceaux grâce aux différentes porteuses:

\begin{figure}[!h]
  \centering
  \includegraphics[width=\textwidth]{Porteuses.png}
  \caption{Porteuses}
\end{figure}

Pour égaliser le signal, il suffit de diviser chaque signal reçu par le gain
associé au canal de propagation estimé.

%%% Local Variables:
%%% mode: latex
%%% TeX-master: "../rapport_de_base"
%%% End:

\chapter{Rôle du préfixe cyclique dans le signal OFDM}

\section{Principe du préfixe cyclique}

Avant de répondre à cette question, nous détaillerons ici le principe d'un
préfixe cyclique et nous expliciterons sa construction. Il est à noter dans un
premier temps qu'un préfixe cyclique est un intervalle de garde
particulier. ~\\

\definition{Intervalle de garde}{Un intervalle de garde est un signal de durée
  $\Delta$ que l'on place avant chaque symboles que nous souhaitons pour réduire
  l'interférence inter-symboles (voir section \ref{ISI}). Deux types
  d'intervalles de garde sont couramment utilisés : le préfixe cyclique et le
  bourrage de zéros.}

Le préfixe cyclique se place donc avant le symbole que l'on souhaite
transmettre. De plus le préfixe cyclique se construit comme la répétition des
derniers échantillons du bloc qu'il précède. C'est-à-dire que si nous souhaitons
transmettre $N$ état alors nous copierons $N_g$ état finaux du
symbole au début. La figure \ref{fig:PC} illustre cette copie.

\begin{figure}[!h]
  \centering
  \includegraphics[width=0.7\textwidth]{pc}
  \caption{Préfixe cyclique}
  \label{fig:PC}
\end{figure}



\section{Rôle du préfixe cyclique}

\subsection{Interférence inter-symboles (ISI)}
\label{ISI}

\definition{Interférence inter-symboles}{\og En télécommunications, une
  interférence inter-symbole est une forme de distorsion d'un signal qui a pour
  effet que le symbole transmis auparavant affecte le symbole aujourd'hui
  reçu\fg{}\cite{def}}


Le préfixe cyclique étant un intervalle de garde permet de se prémunir des
interférences entre symboles (ISI).

Les symboles que nous envoyons subissent des échos. Les échos correspondent aux
signal initialement envoyé mais atténué et retardé. Ils se superposent au signal
reçut de tel façon qu'a un instant $t$ il est possible de recevoir à la foi par le
signal principal le symbole $Si$ et par l'écho le symbole $S_{i-1}$: c'est l'ISI.

Si on suppose connu le temps $T_{max}$ maximal d'un écho (en pratique il est possible de
déterminer les propriétés du canal), et qu'on émet un intervalle de garde
pendant un temps $\Delta > T_{max}$ alors on recevra entre $\Delta$ et
$T_s+\Delta$ uniquement le symbole $S_i$ et l'intervalle de garde qui est
connue. Une illustration est présenté à la figure \ref{fig:intervalleGarde}.
~\\

Il n'y a donc plus d'ISI, on est capable d'extraire facilement l'information.


\begin{figure}[!h]
  \centering
  \includegraphics[width=\textwidth]{IntervalleGarde}
  \caption{Intervalle de Garde}
  \label{fig:intervalleGarde}
\end{figure}

\subsection{Interférence entre porteuses (ICI)}

\definition{Interférence entre porteuses}{Interférence dut au recouvrement des
  sous-porteuse en OFDM}

Dans ce cas ci c'est bien le caractère cyclique du préfixe qui permet d'éliminer
l'interférence entre porteuses (ICI).



%%% Local Variables:
%%% mode: latex
%%% TeX-master: "../rapport_de_base"
%%% End:


\chapter{Justification des relations d'égalisation DFVE}

Dans le document \cite{sujet}, plusieurs équations méritent d'être discutées.

\section{Équation (2) et (4)}
\label{sec:ISI}

\textbf{Énonce} :

Soit $x_i$ et $y_i$ les vecteurs formés des N échantillons des $j^{ème}$
symbole émis et reçus, $b_j$ le vecteur de bruit et $H_0$ et $H_1$ les matrices
triangulaires définies par :

$H_0=
\begin{bmatrix}
  h_0 & 0 & \cdots & \cdots & \cdots & 0 \\
  h_1 & h_0 & 0 & \cdots  & \cdots & \vdots \\
  \vdots & & \ddots & \ddots &  & \vdots \\
  h_{N-1} & h_{N-2} & \cdots & \cdots & h_1 & h_0

\end{bmatrix}
$  $H_1=
\begin{bmatrix}
  0 & h_{N-1} & h_{N-2} & \cdots & \cdots & h_1 \\
  0 & 0 & h_{N-1} & h_{N-2} &  & h_2 \\
  \vdots & 0 & \ddots & \ddots &   & \vdots \\
  \vdots &  &  & \ddots  & \ddots & h_{N-1} \\
  0 & \cdots & \cdots & \cdots & 0 & 0 \\


\end{bmatrix}
$

Alors,

$ y_j = H_0*x_j+H_1*x_j-1+b_j$
~\\

Et en FFT, on peut écrire:

$ Y_j = C_{dft}(H_0*C_{DFT}^{-1}*X_j + H_1*C_{DFT}^{-1}*X_{j-1}) + B_j$


\section{Équation (6) et (8)}

\textbf{Énonce : }

L'égalisation doit fournir une estimation de $\widehat{X_j}$. Dans le domaine
fréquentiel l'équation est:
~\\

$\widehat{X_j}= P_0*Y_j+P_1*\tilde{X_j}$

Les matrices $P_0$ et $P_1$ suivant le critère de ZF (Zero forcing) sont:

$P_0^{ZF} = C_{DFT}*H_0^{-1}*C_{DFT}^{-1}$

$P_1^{ZF} = - C_{DFT}*H_0^{-1}*H_1*C_{DFT}^{-1}$




%%% Local Variables:
%%% mode: latex
%%% TeX-master: "../rapport_de_base"
%%% End:

\chapter{Tests0 des structures d'égalisation DFVE}

\paragraph{}
Nous devons vérifier expérimentalement deux structures d'égalisation DFVE
(temporelle et fréquentielle). Dans ce chapitre, nous aurons connaissance du
canal de propagation à la réception, sans l'utilisation des pilotes. Nous avons
donc crée un signal OFDM, et modélisé l'effet du canal de propagation. Ensuite,
nous avons testé les deux structures d'égalisation DFVE en connaissant la
réponse du canal, et donc sans algorithme d'estimation des coefficients
complexes (Chapitre suivant). Les codes MATLAB commentés sont disponibles avec ce
rapport.

\section{Création du signal}

\paragraph{}
Nous avons choisi de prendre 4 sous-porteuses avec la première à 2.412 GHz, et
les suivantes espacée de 0.3125MHz afin du simuler 4 sous-porteuses du canal 1
du WIFI en France. Nous avons choisi comme modulation, une $\pi/4$-DPSK dont les
états représentent les chiffres 1, 2, 3 et 4. Nous créerons aléatoirement
un vecteur de ces 4 chiffres, et le modulerons. Nous pouvons voir le diagramme de
constellation sur la Figure ~\ref{QPSK}.

\paragraph{}
\vspace{1\baselineskip}
\begin{figure}[!h]
  \centering
  \includegraphics[scale=0.6]{QPSKdep.png}
  \caption{Diagramme de constellation avant l'IFFT }
	\label{QPSK}
\end{figure}
\vspace{10\baselineskip}

\paragraph{}
Ensuite, nous créerons notre signal après en avoir effectué l'IFFT. Nous obtenons le signal visible sur la Figure ~\ref{signal}.

\paragraph{}
\vspace{1\baselineskip}
\begin{figure}[!h]
  \centering
  \includegraphics[scale=0.6]{signalSR.png}
  \caption{Signal à la sortie du récepteur }
	\label{signal}
\end{figure}
\vspace{1\baselineskip}

\section{Canal de propagation}
\paragraph{}
Nous avons modélisé la réponse fréquentielle du canal par le filtre d'un canal écho de fonction de transfert $H(f)=1+(0.4+j*0.2)*f^{-1}$. Sur la Figure ~\ref{fonctionT}, nous pouvons voir la réponse en gain et en phase du canal.
\paragraph{}
\vspace{1\baselineskip}
\begin{figure}[!h]
  \centering
  \includegraphics[scale=0.8]{fonctionT.png}
  \caption{Réponse fréquentielle du canal sur la bande du signal OFDM}
	\label{fonctionT}
\end{figure}
\vspace{1\baselineskip}
\paragraph{}
Ensuite nous ajoutons notre signal temporelle à la réponse fréquentielle du
canal par convolution. Puis, nous ajoutons un bruit Gaussien complexe de variance
0.01. Notre signal est ainsi transformé comme nous pouvons le voir sur la Figure
~\ref{ajoutCanal}
\paragraph{}
\vspace{1\baselineskip}
\begin{figure}[!h]
  \centering
  \includegraphics[scale=0.8]{ajoutCanal.png}
  \caption{Signal à l'entrée du récepteur sans bruit et avec bruit}
	\label{ajoutCanal}
\end{figure}
\vspace{1\baselineskip}


\section{Structure fréquentielle DFVE}

\paragraph{}
Pour l'égalisation, nous avons un traitement post-réception afin de compenser
les effets du canal de propagation. Ici, nous avons connaissance de nos
coefficients des matrices $H_0$ et $H_1$ \cite{sujet}, nous n'appliquons donc pas
l'algorithme LMS. Grâce aux deux matrices précédentes, nous avons calculé les
matrices $P_0$ et $P_1$ de la Figure ~\ref{dfveF}.

\paragraph{}
\vspace{1\baselineskip}
\begin{figure}[!h]
  \centering
  \includegraphics[scale=1]{dfveF.png}
  \caption{Chaîne de réception de la structure fréquentielle DFVE}
	\label{dfveF}
\end{figure}
\vspace{1\baselineskip}
\section{Structure temporelle DFVE}

\paragraph{}
Après notre traitement, on peut voir sur la Figure ~\ref{resultDFVEF}, que les
états du signal à la réception après égalisation correspondent bien aux
constellations du signal de départ. De plus, après démodulation de ces états,
nous avons comparé nos résultats au signal de départ. Pour un faible bruit blanc
Gaussien complexe (variance de 0.01) dans le canal de propagation, nous avons
une erreur de 0\%. Pour un bruit de variance 0.05, nous avons 5\% d'erreurs.

\paragraph{}
\vspace{1\baselineskip}
\begin{figure}[!h]
  \centering
  \includegraphics[scale=0.8]{resultDFVEF.png}
  \caption{Diagramme de constellation après égalisation à la réception}
	\label{resultDFVEF}
\end{figure}
\vspace{1\baselineskip}

\section{Structure temporelle DFVE}
\paragraph{}
Avec la structure d'égalisation temporelle DFVE ~\ref{dfveT}, nous avons
exactement les mêmes résultats. Ce qui montre qu'en connaissant exactement la
réponse fréquentielle du canal, les deux structures sont équivalentes. Dans
cette situation, l'avantage de la structure temporelle DFVE est que les matrices
$Q_0$ et $Q_1$ sont triangulaires, ce qui divise par deux la complexité calculatoire.
\paragraph{}
\vspace{1\baselineskip}
\begin{figure}[!h]
  \centering
  \includegraphics[scale=1]{dfveT.png}
  \caption{Chaîne de réception de la structure temporelle DFVE}
	\label{dfveT}
\end{figure}
\vspace{2\baselineskip}

\paragraph{}
Rappelons, avant de passer à la partie suivante, que nos deux structures sont
équivalentes en résultat car nous avons supposé connaître la réponse
fréquentielle du canal. Mais cela n'est pas vraie en pratique. Le but de la
partie suivante sera de tester une méthode d'estimation des coefficients
complexes des matrices intervenant dans l'égalisation DFVE.

%%% Local Variables:
%%% mode: latex
%%% TeX-master: "../rapport_de_base"
%%% End:

\chapter{Estimation de la réponse fréquentielle du canal de propagation dans les structures d'égalisation DFVE}

\paragraph{}
Dans ce chapitre, nous supposons ne plus connaître la réponse fréquentielle du canal, et donc nous ne connaissons plus les matrices qui interviennent dans les structures d'égalisation DFVE. Le but est de les approcher grâce au principe de l'algorithme LMS (Algorithme du gradient stochatisque).

\section{Création du signal et modélisation du canal de propagation}
\paragraph{}
Dans notre signal, nous avons ajouté des pilotes. Les pilotes sont des états connus qui servent à estimer le canal à la réception. Dans notre cas, l'état des pilotes a été pris à $1+j$, et ces pilotes ont été insérer tout les 3 symboles OFDM, sur nos 4 porteuses. Figure ~\ref{etatavecpilotes}, nous pouvons voir le signal que nous voulons envoyer après modulation $\pi/4$-QPSK avec les pilotes. 
\paragraph{}
\vspace{1\baselineskip}
\begin{figure}[!h]
  \centering
  \includegraphics[scale=0.6]{emissionavecpilote.png}
  \caption{Signal après modulation QPSK et avec ajout des pilotes }
	\label{etatavecpilotes} 
\end{figure}
\vspace{10\baselineskip}

\paragraph{}
Le canal de propagation fût modélisé comme dans le chapitre précédent, et Figure ~\ref{sanstraitement}, on peut voir le signal reçu à la réception avant les traîtements.
\paragraph{}
\vspace{1\baselineskip}
\begin{figure}[!h]
  \centering
  \includegraphics[scale=0.7]{sanstraitement.png}
  \caption{Signal reçu par le récepteur}
	\label{sanstraitement} 
\end{figure}
\vspace{2\baselineskip}


\section{Egalisation avec estimation de la réponse du canal}

\paragraph{}
Dans les deux structures, il falait un point de départ pour les matrices. Pour simuler la convergence vers nos bonnes matrices, nous avons fausser celle de départ en ajoutant des valeurs aux coefficients de $H0$ et de $H1$. De plus, nous avons un pas d'adaptation $\mu$ de 0.01. De notre interpretation, le pas d'adaptation petit ralentit l'estimation de la réponse du canal, mais on sera plus précis. Si notre pas est grand, on sera rapide, mais moins précis.

\subsection{Structure fréquentielle d'égalisation DFVE}
\paragraph{}
Dans notre chaîne de réception, lors de la réception de nos pilotes, nous réestimons nos matrices $P0$ et $P1$ avec les formules 12a et 12b de \cite{sujet}. Une fois les états du signal estimés, nous enlevons les pilotes, et nous obtenons le diagramme de constellation visible sur la Figure ~\ref{etatsanspilote}.

\paragraph{}
\vspace{1\baselineskip}
\begin{figure}[!h]
  \centering
  \includegraphics[scale=0.6]{etatsanspilotes.png}
  \caption{Etats du signal estimés après extraction des pilotes dans la structure fréquentielle }
	\label{etatsanspilote} 
\end{figure}

\paragraph{}
Après comparaison avec le signal emis, nous obtenons 0\% d'erreurs sur l'ensemble du signal.


\subsection{Structure temporelle d'égalisation DFVE}

\paragraph{}
Pour la structure temporelle, on estime Q0 et Q1 dans la boucle, mais on fait une estimation sous optimale sur Q1 afin de réduire la complexité. A cause de cette estimation, nous avons toujours plus d'erreurs que dans la structure fréquentielle, comme 1.12\% où l'on trouvait 0\% pour la structure fréquentielle.
\paragraph{}
Afin de mieux illustrer le signal après estimation, nous pouvons voir sur la Figure ~\ref{avecPilote} l'estimation des états par la structure temporelle, avant extraction des pilotes.
\paragraph{}
\paragraph{}
\vspace{1\baselineskip}
\begin{figure}[!h]
  \centering
  \includegraphics[scale=0.6]{avecPilote.png}
  \caption{Etats du signal estimés dans la structure temporelle avant extraction des pilotes }
	\label{avecPilote} 
\end{figure}
\vspace{1\baselineskip}
\paragraph{}
Afin de tester la convergence de notre méthode, nous avons tester un code où nous n'ajoutons pas de bruit complexe dans le canal. Nous mettons beaucoup d'erreurs dans les matrices H0 et H1 initiales, donc dans Q0, Q1, P0 et P1. Dans un premier test, nous gardons cette matrice, et ne la régulons pas. Nous obtenons 26\% d'erreurs dans les deux structures. Ensuite nous relançons l'algorithme avec l'estimation continue des matrices. Nous avons 19\% d'erreurs sur la structure fréquentielle, et 20\% sur la structure temporelle. C'est normal d'avoir un fort taux d'erreurs ici, car il faut le temps que le récepteur converge vers la bonne estimation. Ce qui est important est que nous voyons que nous sommes meilleurs avec la convergence vers la bonne solution grâce à l'algorithme LMS, car on fini par bien estimer notre canal. C'est ce que l'on peut voir sur la Figure ~\ref{convergence}. On commence par des estimations erronnées, et à la fin, on peut voir la convergence vers les bons états.
\paragraph{}
\vspace{1\baselineskip}
\begin{figure}[!h]
  \centering
  \includegraphics[scale=0.5]{convergence.png}
  \caption{Convergence vers les bons états grâce à l'algorithme LMS }
	\label{convergence} 
\end{figure}
\paragraph{}
Nous venons de tester les estimations de la réponse du canal de propagation à travers les estimations de matrices. Nous avons constater un compromis entre complexité de calcul et efficacité de l'éstimation. En effet, la structure temporelle diminue fortement la complexité calculatoire, mais il y a plus d'approximation dans l'éstimation du canal, ce qui nous fait perdre de l'éfficacité dans l'éstimation final du signal émis. Il faut donc trouver un juste milieu dans ce que l'on peut corriger grâce à un codage canal, et la puissance et le temps de calcul minimum.







%%%% CONCLUSION %%%%%%%%%

\chapter*{Conclusion}
\addcontentsline{toc}{chapter}{Conclusion}
\newpage

%%%% ANNEXE %%%%%%%%%%%%

%\part*{Annexe}
\appendix
\nocite{*}
%\input{annexe_}
\newpage
 \listoffigures
 \printindex
 \bibliographystyle{frplain}
  \bibliography{biblio}

\end{document}
%%%%%%%%%%%%%%%%% FIN DU DOCUMENT
%%% Local Variables:
%%% mode: latex
%%% TeX-master: t
%%% End:
